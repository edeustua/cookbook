\documentclass{article}

% Package section
\usepackage[english]{babel}
\usepackage[utf8]{inputenc}
\usepackage[T1]{fontenc}
\usepackage{gensymb}
\usepackage[margin=1in]{geometry}
\usepackage[sc]{mathpazo}

\usepackage{cuisine}

% From old version
%\documentclass[letterpaper,fontsize=11pt]{scrartcl}
%\usepackage[tmargin=0.8in,bmargin=0.8in,lmargin=0.8in,rmargin=0.8in]{geometry}
%\usepackage{xcolor}
%\usepackage[utf8]{inputenc}
%\usepackage[english]{babel}
%\usepackage{hyperref}
%\usepackage{cuisine}

\title{\bfseries E \& L Cookbook}
\author{Laura and Emiliano}


% Styling
\renewcommand*{\recipetitlefont}{\large\bfseries\sffamily}
%\renewcommand*{\recipenumberfont}{\large\bfseries\sffamily}
%\renewcommand*{\recipequantityfont}{\sffamily\bfseries}
%\renewcommand*{\recipeunitfont}{\sffamily}
%\renewcommand*{\recipeingredientfont}{\sffamily}
%\renewcommand*{\recipefreeformfont}{\itshape}

\begin{document}
\maketitle

\begin{recipe}{Empanadas de carne}{8 Portions}{3 hours}
  \ingredient[1]{kg}{flour}
  \ingredient[120]{g}{lard}
  \ingredient[20]{g}{salt}
  \ingredient[\fr12]{L}{water}
  Mix flour with melted lard and add water while mixing the dough. Make sure that
  there is water enough for a good smooth dough. Cannot be sticky. Let rest in
  fridge to relax elasticity (for about 1 hour).

\freeform
  Add the stuffing inside and close the dough by wetting one's finger with
  water. Then bake at 400F or deep fry.

%  Add anything inside and close using water. Then bake at 400F or deep fry.
%
%\freeform\hrulefill
\end{recipe}

\begin{recipe}{Suby's Lentils}{4 portions}{1 Hour}
\ingredient[4]{cups}{Rinsed Lentils}
\ingredient[2]{cloves}{Garlic}
\ingredient[1]{}{Onion}
\ingredient[]{}{Ginger}
\ingredient[]{}{Turmeric}
\ingredient[to taste]{}{Salt and peper}
\ingredient[]{}{Olive oil}
\ingredient[]{}{Water}
    
Heat the olive oil in a large pot
Saute Onion and Garlic
Add Ginger and Turmeric
Add Lentils
Add Water, salt and pepper
Let it simmer. Lower the heat and cook for about 1 hour.

\end{recipe}

\begin{recipe}{Pan de bono}{2 Portions}{\fr12 hour}
  \ingredient[1 \fr14]{cups}{tapioca flour}
  \ingredient[\fr12]{tsp}{baking powder}
  \ingredient[\fr14]{tsp}{salt}
  Mix all of the dry ingredients in a food processor (or in a bowl).

  \ingredient[10]{oz}{queso fresco}
  \ingredient[8]{tbsp}{butter}
  \ingredient[1]{}{egg}
  Add the butter (without melting), one egg and the cheese to the food
  processor. Mix thoroughly.

  \freeform
  Once the butter, egg and cheese are added, the mixing with the food processor
  produces a powder-like dough. Keep mixing until it becomes a solid dough. Add
  a tbsp of water/milk if it is too dry.
 Set the oven to 450\degree{}F and bake for 15 min or until crust forms and
 browns a little bit.%°F.
\end{recipe}

\begin{recipe}{Pastel de acelga}{4 Portions}{2 hours}
  \ingredient[2 \fr12]{cups}{all-purpose flour}
  \ingredient[\fr12]{cup}{water}
  \ingredient[\fr12]{cup}{lard}
  \ingredient[\fr14]{tsp}{salt}
  \ingredient[1]{tsp}{sugar}
  \ingredient[5--6]{tbsp}{cold water}
  In a food processor combine (do not melt the butter and lard) flour, butter,
  lard, salt and sugar. Pulse the mixture until it becomes a powder-like dough.
  It is very probable that it will not. The desired dough is as hard as possible
  (meaning very lowly hydrated) but needs to be compact and resistant to
  tearing. Add cold water to achieve these properties. Once the dough is ready,
  put in a bowl, cover with plastic, and put in the fridge for 20 minutes.

  \ingredient[2]{bunches}{green swiss chard}
  \ingredient[1]{bunch}{spinach}
  \ingredient[2]{cloves}{garlic}
  \ingredient[1]{}{white/yellow onion}
  \ingredient[1]{pinch}{nutmeg}
  \ingredient[1]{tbsp}{aji amarillo}
  \ingredient[1]{}{egg}
  \ingredient[\fr34]{cup}{Parmesan cheese}
  \ingredient[]{}{olive oil}
  \ingredient[]{}{salt}
  \ingredient[]{}{pepper}
  Chop onions and garlic finely and saut\'{e} them until
  translucent in a large pot with olive oil. Keep the cooked onions and garlic aside. Next,
  remove the stems from the swiss chard and, optional, the spinach. Cook the
  swiss chard and spinach in a large pot until completely reduced. Let them cool
  down, and then proceed to squeeze the liquid off the cooked leaves building
  small ball ins the process. Next, get rid of all the water. Once this is done,
  mix the onions and garlic with the greens and add one egg and the shredded
  parmesan cheese. Add salt and pepper to taste, and finally the nutmeg. Mix thoroughly.

  \ingredient[5]{}{eggs}
  \ingredient[]{}{limes}
  Take the dough from the fridge and roll it with a rolling pin until flat and
  approximately 5 mm thick. Make sure you reserve some dough for the top crust.
  This will be the bottom crust. Make sure it fits
  in a baking tray (including the sides). Once it is in the baking tray, use a
  fork to make little indentations on the bottom.

  Next, fill the preparation of step 2 in it. Make 4 holes (depressions) with
  a spoon in the filling, once per quadrant, to put an eggs in each one. Finally, make
  the top crust by using the dough left aside and cover the filling. Make sure
  that the top crust is ``glued'' using water against the rest of the assembly.

  Once the assembled pie is done, make egg wash with the last egg. Apply it over
  the top crust with a brush.

  \freeform
  Preheat the oven to 390\degrees{}F. Bake the pie for approximately 45--55
  minutes, or until the top crust is golden.

  Serve cold with the half of a lime.
  
\end{recipe}

\begin{recipe}{Roberta's pizza dough}{1 Large pizza}{24 hours}
  \ingredient[153]{g}{00 (or bread) flour}
  \ingredient[153]{g}{all purpose flour}
  \ingredient[200]{g}{water}
  \ingredient[8]{g}{salt}
  \ingredient[2]{g}{active dry yeast}
  \ingredient[4]{g}{extra virgin olive oil}
  \ingredient[1]{tsp}{sugar}
  To activate the yeast, warm up the water to approximately 45\degrees{}C, and
  mix it with the sugar and yeast. Leave it still for 10 minutes.

  In a bowl, put the sifted flours and the salt and mix them. Once the yeast is
  activated, add it to the bowl with the olive oil. Knead by hand by
  approximately 30 minutes, until the dough is smooth and homogeneous. This will
  feel less sticky to the hand and also will fill less elastic than at the beginning.

  \freeform
  Once the dough is done, put in round containers (round tuppers work best) with
  olive oil to prevent the dough from sticking. Leave in the fridge for at least
  24 hours. Best results can be achieved after 3 days of rising.

  Take the dough out from the fridge 1 hour before baking the pizza. Stretch the
  dough to fill the pizza tray and put in the toppings. Bake at 500\degress{}F
  for 15 minutes.
  

\end{recipe}

\begin{recipe}{Bechamel sauce}{1 Lasagna}{\fr12 hour}
  \ingredient[2]{cloves}{garlic}
  \ingredient[4]{cups}{milk}
  \ingredient[4]{tbsp}{butter}
  \ingredient[\fr3/4]{cups}{all-purpose flour}
  \ingredient[1]{pinch}{nutmeg}
  \ingredient[]{}{salt}
  \ingredient[]{}{(white) pepper}
  Chop the garlic and saut\'{e} it in a sauce pan with butter. Add milk and wait
  until it warms up. Next, start adding the flour, \fr12 of a cup first, and then the
  rest, slowly, until the sauce thickens. Finally, add the nutmeg, salt, and
  pepper to taste.


\end{recipe}

\end{document}
